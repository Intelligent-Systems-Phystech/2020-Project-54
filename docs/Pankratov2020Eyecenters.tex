
\documentclass[12pt,a4paper,notitlepage]{article}

\usepackage[T2A]{fontenc}
\usepackage{jmlda}
\usepackage[english,russian]{babel}
\usepackage{amssymb,amsfonts,amsmath,mathtext,cite,enumerate,float}

\begin{document}
 \bibliographystyle{abbrv}
\nocite{*}

\title{Поиск зрачка на изображении глаза методом проекций яркости.}
\author{В.\,В.~Панкратов} 
\abstract{Изучается проблема быстрого распознавания зрачка на изображении. Мы работаем с изображением как с  функцией яркости от пары 
аргументов. Чтобы улучшить скорость ее обработки, предполагается применение метода проекции,который  использует функцию проекции 
интенсивности на горизонтальные и вертикальные оси и используя  ее строит положение центра глаза. Предполагается модифицировать и исследовать как можно оптимизировать этот алгоритм,используя нейросети.
}
\bigskip


\maketitle


\section{Введение}
	Одна из задач,которую ставит перед собой обработка компьбтерных изображений - определение лица человека на фотографии. Такая формулировка вносит неточность,поэтому принято определять положения лица путем его внешне заметных черт,одной из которых является центр зрачка.Эту задачу решает,например, статья[2],однако нас интересует более простой и быстрый алгоритм. В статье[1] предложен быстрый алгоритм  нахождения центра,который определяет верхнюю нижнюю правую и левую "границы" глаза путем построения функции проекции ,определенной в [3]. В данной работе наша цель модифицировать этот алгоритм, изменив положение осей, построив проекции на них, и наблюдая за изменением результатов. Также мы применим нейросеть для этой задачи и определим, какой вклад она окажет на результат работы предложенного алгоритма. 
\section{Постановка задачи}
Дано цифровое изображение лица. Предполагается, что оно черно-белое, то есть каждой пиксель определяется функцией яркости $I(i,j)$.Проводится сокращение размерности данных до одномерной картины, используя функции проекции(для простоты укажем их определение для горизонтальной и вертикальной проекций).	
$$P_H(j) =\frac{1}{i_M -i_m +1} \sum_{i=i_m}^{i=i_M} I(i,j) $$
$$P_V(i) =\frac{1}{j_N -j_n +1} \sum_{j=j_n}^{j=j_N} I(i,j) $$
Требуется решить задачу нахождения центра глаза по этим проекциям.
\par Пусть дана тренировочная выборка $\mathcal{D}$ объектов и ответов:$(\overline{x_i},y_i)$. Будем рассматривать отдельно задачу для левого и правого глаза.Необходимо определить оптимальную функцию $f$,дающую оптимальный ответ на $\mathcal{D}$ , то есть решить задачу минимизации функционала качества:
$$Q(f,\mathcal{D}) = \frac{1}{|\mathcal{D}|}\sum \mathcal{L}(\overline{x},y)$$
Функционал качества отсюда определим как сумму квадратов расстояний до действительных их координат:
$$Q(f,\mathcal{D}) = \sum_{\mathcal{D}}(y_i - f(\overline{x_i}))^2 = \sum_{\mathcal{D}}\varepsilon^2$$
Получаем задачу оптимизации
$$f^* =  \underset{f}{\arg \min} ~Q(f,\mathcal{D})$$	
Ее можно решить при помощи градиентного спуска
$$ -dQ = \sum(y_i - f(\overline{x_i}))df $$
При определении функции ошибки на тесте учтем, что качество необходимо оценить по результатам для обоих глаз. На тестовой выборке таким образом ошибка определяется как:
$$\epsilon = \frac{\max (\varepsilon_L,\varepsilon_R)}{D_{LR}}$$
Здесь $D_{LR}$ -- расстояние между центрами глаз, а $\varepsilon_{L,R}$ -- $\varepsilon$ левого или правого глаза соответственно.
\bibliography{11} 


 

\end{document}
